%% Example of a presentation

\section{Basic frames}

\subsection{Elements}

\frame{
\frametitle{[TITLE]}
\framesubtitle{[SUBTITLE]}
\begin{columns}
\column{0.5\textwidth}
COLUMN 1
\column{0.5\textwidth}
COLUMN 2
\end{columns}
\begin{center}
some text
\note{This is a note, not visible by default in the presentation}
\end{center}
}

\frame{
	\frametitle{Blocks}
	\begin{block}{[block]} ABC \end{block}
	\begin{exampleblock}{[exampleblock]} ABC \end{exampleblock}
	\begin{alertblock}{[alertblock]} ABC \end{alertblock}
}

%\beamertemplatesolidbackgroundcolor{white}
\beamertemplateshadingbackground{yellow!50}{white!50}
\frame{
	\frametitle{Highlighting and background colors}
	Normal, \alert{alert}, \structure{structure}.
	\par And a box:
	\setlength{\fboxrule}{2pt}
	\setlength{\fboxsep}{2pt}
	\fcolorbox{blue}{green!25}{
		\textcolor{gray}{
			Some text...
		}
	}
}
\beamertemplateshadingbackground{white}{white}

\subsection{Options}

\frame[plain,t]{
	\frametitle{Plain frame}
	\texttt{plain} frame removes header, footers, etc.
	to leave as much space as possible
	\par
	Also, contents of this frame are aligned to the top by giving an option
	\texttt{t}.
}

\frame[allowframebreaks]{
	\frametitle{With frame breaks}
	\texttt{allowframebreaks} will split the contents of a frame across many
	slides if it needs more space
}
\frame[shrink]{
	\frametitle{Shrunk frame}
	\texttt{shrink} option means that the contents will be shrunk if there is
	not enough space on the frame
	\vspace{5cm}
	some text;
	here vertical space was added to force the shrinking
	\vspace{5cm}
}

\subsection{Other}

\frame{
	\frametitle{References}
	A simple reference to Wikipedia \cite{wikipedia}.
}

\section{Floating elements}

\subsection{Images}

\frame{
	\frametitle{Image}
	\centering
	\includeimage[width=0.5\linewidth,height=4cm]{logo}
}

\frame{
	\frametitle{TikZ}
	\begin{center}
	\begin{tikzpicture}
	\begin{axis}
	\addplot+[sharp plot] coordinates
	{(0,0) (1,2) (2,3)};
	\end{axis}
	\end{tikzpicture}
	\end{center}
}

\subsection{Tables and minipages}

\frame{
	\frametitle{Tables and minipages}
	\begin{columns}
	\column{0.4\textwidth}
	\begin{table}[H]
	\centering
	\rowcolors{1}{blue!20}{blue!5}
	\begin{tabular}{|l|c|}
	\hline
	hello & world \\
	abc & def \\
	ghi & jkl \\
	mno & pqr \\
	\hline
	\end{tabular}
	\caption{Colored table}
	\end{table}
	\column{0.4\textwidth}
	\fbox{\begin{minipage}{\linewidth}
	This is a minipage.
	\end{minipage}}
	\end{columns}
}

\subsection{Listings}

\defverbatim[colored]\listingCHelloWorld{%
\begin{lstlisting}[style=c,title={C Hello World}]
#include <stdio.h>

int main(void) {
	printf("hello world\n");
	return 0;
}
\end{lstlisting}}
\frame{
	\frametitle{Listings}
	\listingCHelloWorld
}

\section{Animations}

\frame{
	\frametitle{Pausing}
	\texttt{\textbackslash{}pause} can be added nearly anywhere to stop drawing
	at that point and continue later
	\pause
	\par
	...that's nice
}

\frame{
	\frametitle{Slide transition}
	\texttt{
		\only<1>{...........}
		\only<2>{hello world}
		\only<3>{other words}
	}
	\transwipe<1-2>[direction=0]
	\transduration<1-2>{2}
	\transdissolve<3>
}

\frame{
	\frametitle{Some lists}
	\begin{itemize}
	\item<+-> point 1
	\item<+-> point 2
	\item<+-> point 3
	\end{itemize}
	\transdissolve
}

\frame{
	\frametitle{Highlighting bullets}
	\framesubtitle{Using \texttt{\textbackslash{}alt}}
	This is done using \texttt{%
		\textbackslash{}alt\textless{}overlay\textgreater%
		\{in overlay\}\{out of overlay\}}.
	\begin{itemize}
	\item<1-> \alt<2>{\color{blue}}{\color{gray}} point one
	\item<1-> \alt<3>{\color{blue} point two}{\color{gray} point ---}
	\end{itemize}
}

\def\hilite<#1>{\temporal<#1>%
{\color{gray}}{\color{blue}}{\color{black}}%
}
\frame{
	\frametitle{Other highlighting}
	\framesubtitle{Using \texttt{\textbackslash{}temporal}}
	\begin{itemize}
	\hilite<2>\item one
	\hilite<3>\item two
	\hilite<4>\item three
	\end{itemize}
}
