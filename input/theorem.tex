%% theorem

\usepackage{amsthm}

%% Usage
% \newtheorem[*]{name}[counter]{text}[section]
% \theoremstyle{name}...
% ifthenelse is used to avoid conflicts with Beamer

\theoremstyle{plain}
\ifthenelse{\isundefined{\lemma}}{
\newtheorem{lemma}{\lemmaname}
}{}
\newtheorem{proposition}{\propositionname}
\ifthenelse{\isundefined{\theorem}}{
\newtheorem{theorem}{\theoremname}
}{}

\theoremstyle{definition}
\ifthenelse{\isundefined{\definition}}{
\newtheorem{definition}{\definitionname}
}{}
\newtheorem{property}{\propertyname}

\theoremstyle{remark}
%\newtheorem{example}{\examplename}
%\newtheorem{note}{\notename}
\newtheorem{remark}{\remarkname}
